\title{Math 239 Fall 2023 Tutorial Answers Week 5}
\date{2023 Oct. 19/20}
\maketitle

\begin{enumerate}
    \question{A Routine Recurrence} We use theorems $3.2.1$ and $3.2.2$ from the IC (Introduction to Combinatorics notes) on Learn. If one is not familiar with the notation or the theorems, one should read over them.

    The characteristic polynomial of this recurrence is
    \begin{align*}
        x^3 + 5x^2 + 3x - 9 = (x-1) (x+3)^2.
    \end{align*}
    By theorem $3.2.2$ then this implies that we have the closed form
    \begin{align*}
        a_n = A \cdot (1)^n + (B + Cn) \cdot (-3)^n
    \end{align*}
    for some constants $A,B,C$. Plugging in initial values of $a_n$ gives us
    \begin{align*}
        a_0 &= A + B\\
        a_1 &= A - 3B - 3C\\
        a_2 &= A + 9B + 18C.
    \end{align*}
    Solving this system of linear equations gives us
    \begin{align*}
        A &= 2,\\
        B &= -1,\\
        C &= \frac{2}{3}.
    \end{align*}
    This implies that
    \begin{align*}
        a_n = 2 + (-1 + \tfrac{2}{3}n)(-3)^n.
    \end{align*}
    Multiplying the initial conditions by $3$ just scales $A,B,C$ by $3$, and so we get that
    \begin{align*}
        a_n &= 6 + (-3 + 2n)(-3)^n\\
        &= 3(2 + (-1 + \tfrac{2}{3}n)(-3)^n).
    \end{align*}

    \newpage
    \question{Another Recurrence} 
    \begin{enumerate}
        \item \[a_n = 3a_{n-1} + 2a_{n-3}.\]
        \item \[a_0 = 1, a_1 = 3, a_2 = 9.\]
        \item The answers varies. One possibility is coloured compositions of $n$ where all parts are $1$ or $3$; each part can be coloured independently red, green, or yellow; and each $3$ part can be coloured independently blue or purple. 
    \end{enumerate}

    \newpage
    \question{Degree Sequences} 
    \begin{enumerate}
     \item $4, 3, 2, 2, 1$ Graph exists, draw one vertex adjacent to the other four, then pick another vertex to be adjacent to two others.  
        \item $6, 5, 4, 3, 2, 1$
        Not possible because there aren't enough vertices for the first vertex to be adjacent to.
        \item $5, 4, 4, 3, 2, 1$
        Not possible because odd number of odd vertices (contradicts handshaking lemma)
        \item $3, 3, 3, 3, 3, 3$
        This is just $K_{3,3}$
        \item $2, 2, \dots, 2$ for arbitrary length $n$ is just cycle of length $n$
        \item $6, 6, 4, 2, 2, 2, 1, 1$ The only way to have two vertices of degree $1$ in this graph is to have the two vertices of degree $6$ adjacent to each other, as otherwise forces all of the other vertices to have degree at least $2$. However, even if the two vertices of degree $6$ are adjacent to each other, we have a graph where two vertices have degree $1$ and every other vertex has degree $2$, so there's no way to add edges to increase the degree of just one vertex to $4$ unless we add a loop. Havel-Hakimi algorithm verifies that this isn't the degree sequence of a graph.
    \end{enumerate}

    \newpage
    \question{Graph Complements Solution} Recall that if $G$ is a graph $\overline{G}$, the complement of $G$, has $V(G)=V(\overline{G}$ and $E(\overline{G}=\{uv| uv\notin E(G), u,v\in V(\overline{G})\}$.
    \begin{enumerate}
        \item Let $K_n$ be the complete graph on $n$ vertices. Describe $\overline{K_n}$.\\
                Solution: $\overline{K_n}$ is the graph of $n$ isolated vertices.
        \item For any $v\in G$ what is $deg_G(v)+deg_{\overline{G}}(v)$?\\
                Solution: $|V(G)|-1$
        \item Find a graph $G$ such that $G\cong\overline{G}$. \\
                Solution: The two simplest example are $P_4$ and $C_5$.
        \item Prove that if $|V(G)|\geq 6$ then either $G$ or $\overline{G}$ contains a triangle.\\
                Solution: Let $v\in V(G)$. Since $\text{deg}_G(v)+\text{deg}_{\overline G}(v)=n-1\ge 5$, either $\text{deg}_G(v)\ge 3$ or $\text{deg}_{\overline G}(v)\ge 3$. Suppose $\text{deg}_G(v)\ge 3$. Let $N_G(v)$ be the set of neighbours of $v$ in $G$. Thus, $|N_G(v)|\ge 3$. If there are two vertices in $N_G(v)$ that are adjacent, then these two vertices together with $v$ form a triangle in $G$. Suppose every pair of vertices in $N_G(v)$ are not adjacent, then all vertices in $N_G(v)$ are pairwise adjacent  in ${\overline G}$, implying that ${\overline G}$ has a triangle, as $|N_G(v)|\ge 3$. The proof for the case  $\text{deg}_{\overline G}(v)\ge 3$ is similar, with $G$ and ${\overline G}$ interchanged. \qed    \end{enumerate}
\end{enumerate}