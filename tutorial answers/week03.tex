\title{Math 239 Fall 2023 Tutorial Answers Week 3}

\date{2023 Sep. 28/29}
\maketitle

\begin{enumerate}
    \question{Ambiguous Expressions} This is fairly easy warm-up question.
    \begin{enumerate}
        \item A string with only even blocks has a natural decomposition given by
        \begin{align*}
            S = \{ 00, 11\}^* = \{ 00\}^* \big( \{ 11\} \{11\}^* \{ 00 \} \{ 00 \}^*  \big)^* \{11\}^*.
        \end{align*}
        \item We use the second one. This gives us
        \begin{align*}
            S(x) = \frac{1}{1-x^2} \cdot \left( \frac{1}{1- x^2 \cdot \frac{1}{1-x^2} \cdot x^2 \cdot \frac{1}{1-x^2}} \right) \cdot \frac{1}{1-x^2} = \frac{1}{1-2x^2},
        \end{align*}
        where the last step follows by Wolfram Alpha. This is not surprising, since we are basically just taking normal binary strings with double the weight.
        \item We use the second one. This gives us
        \begin{align*}
            S(x) = \frac{1}{1-x^2} \cdot \left( \frac{1}{1 - x^4 \cdot \frac{1}{1-x^4} \cdot x^2 \cdot \frac{1}{1-x^2}} \right) \cdot \frac{1}{1-x^4} = \frac{1}{1-x^2-x^4},
        \end{align*}
        where the last step follows by Wolfram Alpha. This is again not surprising.
        \item This should work with any ambiguous representation. When you have a GF from an ambiguous rep, you are overcounting things, so you end up with something like each ambiguous coefficient being larger than each unambiguous coefficient.
    \end{enumerate}

    \newpage
    
    \question{Unambiguous Expressions} 
    \begin{enumerate}
        \item We can construct the regular expression as follows \[\varepsilon \smile 1 \smile 0\smile 1(0\smile 1)^*1\smile 0(0\smile 1)^*0\] and in fact, if ${\rm K}$ is an unambiguous regular expression that produces all possible binary strings, ${\rm R}$ can be chosen to be \[ \varepsilon \smile 1 \smile 0\smile 1{\rm K}1 \smile 0{\rm K}0,\] leading to the rational function \[1+x+x+ x \frac{1}{1-2x} x+x \frac{1}{1-2x}x=1+2x+\frac{2x^2}{1-2x}\,.\]
        \item We can construct the regular expression as follows \[0^*(1000000^*)^*\] leading to the rational function \[\frac{1}{1-x}\frac{1}{1-\frac{x^6}{1-x}}=\frac{1}{1-x-x^6}\,.\]
    \end{enumerate}
    \newpage
    
    \question{Compositions} 
    For different parity of the length of compositions, we have $$\mathcal{E}_{2k+1} = \overbrace{O \times E \times O \times \cdots \times O}^\text{$2k+1$} $$
    $$\mathcal{E}_{2k} =  \underbrace{O \times E \times O \times \cdots \times E}_\text{$2k$}
    $$
    where $E = \{2, 4, 6 ,\dots\}, O = \{1, 3, 5, \dots\}$. Then $$
    \Phi_E(x) = x^2 + x^4 + x^6 + \cdots + = \frac{x^2}{1-x^2}
    $$
    $$
    \Phi_O(x) = x + x^3 + x^5 + \cdots + = \frac{x}{1-x^2}$$
    So we have 
    $$\Phi_{\mathcal{E}_{2k+1}} = \Phi_O(x)^{k+1} \Phi_E(x)^k = \frac{x^{3k+1}}{(1-x^2)^{2k+1}}
    $$
    $$\Phi_{\mathcal{E}_{2k}} = \Phi_O(x)^{k} \Phi_E(x)^k = \frac{x^{3k}}{(1-x^2)^{2k}}
    $$
    \begin{align*}   
    \mathcal{E} &= \bigcup_{k=0}^\infty \mathcal{E}_{2k} \cup \bigcup_{k=0}^\infty \mathcal{E}_{2k+1}\\
    &= \sum_{k=0}^{\infty} \frac{x^{3k+1}}{(1-x^2)^{2k+1}} +  \frac{x^{3k}}{(1-x^2)^{2k}} \\
    &= \sum_{k=0}^{\infty}\frac{x^{3k}}{(1-x^2)^{2k}} (\frac{x}{1-x^2} + 1) \\
    &= \frac{x + 1 - x^2}{1-x^2} \sum_{k=0}^{\infty} \left(\frac{x^3}{(1-x^2)^2}\right)^k
    \\
    &= \frac{1 + x - x^2}{1-x^2} \cdot \frac{1}{1 - \frac{x^3}{(1-x^2)^2}}\\
    &= \frac{(1+x-x^2)(1-x^2)}{(1-x^2)^2 - x^3} \\
    &= 1 + \frac{x}{1-2x^2 - x^3 + x^4}
    \end{align*}
    
    \newpage
    
     \question{Generating Series} We start out by trying to satisfy the requirements: The string starts with $0$, can't have blocks of $0$ so we add a second $0$, and then we can add however many more zeroes we want to that. After that, the sequence has to end with $1$, so maybe we allow any number of $1$'s and then a terminating $1$, which will give us $000^*1^*1$.

     However, this doesn't allow for, say, $001001$, which satisfies the requirements. So we modify our attempt a bit by allowing the pattern to repeat arbitrarily many times, like so: $000^*1^*1(000^*1^*1)^*$. 

     So our unambiguous pattern for our set $S$ is $000^*1^*1(000^*1^*1)^*$. Now we look for the generating function by first finding it for $000^*1^*1$: 

\begin{center}
    $\Phi_{000^*1^*1}(x) = \Phi_{00}(x) \cdot \Phi_{0^*}(x) \cdot \Phi_{1^*}(x) \cdot \Phi_{1}(x) = \Phi_{00}(x) \cdot 
     \frac{1}{1-\Phi_{0}(x)} \cdot \frac{1}{1 - \Phi_{1}(x)} \cdot \Phi_{1}(x) = x^2 \cdot \frac{1}{1-x} \cdot \frac{1}{1 -x} \cdot x = \frac{x^3}{(1 - x)^2}$
\end{center}

Now we plug this in to get:

\begin{center}
    $\Phi_{S}(x) = \Phi_{000^*1^*1}(x) \cdot \Phi_{(000^*1^*1)^*}(x) = \frac{x^3}{(1 - x)^2} \cdot \frac{1}{1 - \frac{x^3}{(1 - x)^2}} = \frac{x^3}{(1 - x)^2} \cdot \frac{(1 - x)^2}{(1 - x)^2 - x^3} = \frac{x^3}{(1-x)^2 - x^3} = \frac{x^3}{(1-x)^2 - 2x + x^2 - x^3} = \frac{x^3}{(1-x)^2 - (2x - x^2 + x^3)}$
\end{center}

We'll expand to find the first 4 terms. Note that we don't need to worry about terms with power 7 or higher, so we just ignore the rest of the sum.

\begin{center}
    $\frac{x^3}{(1-x)^2 - (2x - x^2 + x^3)} = x^3 \sum_{k \ge 0}(2x - x^2 + x^3)^k = x^3 + x^3(2x - x^2 + x^3) + x^3(2x - x^2 +x^3)^2 + x^3(2x - x^2 + x^3)^3 + x^3\sum_{k \ge 4}(2x - x^2 + x^3)^k = x^3 + 2x^4 - x^5 + x^6 + x^3(4x^2 - 2x^3 + 2x^4 - 2x^3 + x^4 - x^5 + 2x^4 - x^5 + x^6) + x^3(8x^3 + \cdots ) + \cdots = x^3 + 2x^4 + 3x^5 + 5x^6 + \cdots$
\end{center}

Check with some examples to verify, e.g. there's only one string of length 3: $001$.

    % \newpage 

    \question{Bonus Material: A Second GF for Binomial Coefficients} This is from Wilf.
    Write
    \begin{align*}
        \sum_n \binom{n}{k} y^n &= [x^k]\sum_n \sum_k \binom{n}{k} x^k y^n = [x^k]\sum_n (1+x)^n y^n\\
        &= [x^k] \frac{1}{1-y(1+x)} = \frac{1}{1-y}[x^k] \frac{1}{1-( \tfrac{y}{1-y} )x}\\
        &= \frac{1}{1-y} \frac{y^k}{(1-y)^k} = \frac{y^k}{(1-y)^{k+1}}.
    \end{align*}
    Pretty nifty!
\end{enumerate}