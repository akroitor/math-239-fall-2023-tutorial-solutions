\title{Math 239 Fall 2023 Tutorial Answers Week 6}

\date{2023 Oct. 26/27}
\maketitle

\begin{enumerate}
    \question{Matching Vertices} Write down the degree sequence. This is a sequence $d_n , \cdots, d_1$ where $0 \leq d_j \leq n-1$.
    
    Suppose that $G$ is connected, so that each vertex has degree at least one. Then $1 \leq d_j \leq n-1$. This means that since there are only $n-1$ possible values for $n$ vertices, there must be some repeated value (by pigeonhole). Thus at least one value in the degree sequence is repeated, and so two vertices have the same degree.

    Now suppose that $G$ has some vertex of degree $0$. Then $0 \leq d_j \leq n-2$. Similarly to before then by pigeonhole there must be two vertices with the same degree.

    \newpage
    \question{Paths and Cycles} Let $G$ be a graph and $k\geq 2$ be the minimum degree of $G$ (denoted $\delta(G)$).
    \begin{itemize}
        \item Prove $G$ has a path of length at least $k$.\\
        Solution: Let $P$ be the longest path in $G$. Let $v_0$ be the first vertex in the path. Observe that every neighbor of $v_0$ is in $P$ or we could build a longer path. Thus $P$ has at least $k+1$ vertices and hence $k$ edges. 
        \item Prove $G$ has a cycle of length at least $k+1$. \\
        Solution: Suppose $P=v_0v_1...v_n$ is the longest path in $G$. Observe that all neighbors of $v_0$ are in the path or we could make the path longer. Let $v_j$ be a neighbor of $v_0$ such that $j$ is as large as possible. That is to say that $v_j$ is the $k$th neighbor we encounter of $v_0$ along the path. Then $Pv_jv_0$, which means taking the path $P$ from the $v_0$ to the vertex $v_j$ and then returning to $v_0$ is a cycle of length $k+1$.  
    \end{itemize}
    

    \newpage
    \question{Vertices and Edges} Assume, for contradiction, that \(G\) can have a vertex of degree \(0\) given the above conditions. This means that within \(G\), there are at least two components: one with a single vertex and no edges, and one with \(n - 1\) vertices and \(m > \frac{(n-1)(n-2)}{2}\) edges. We can attempt to construct the component with \(n-1\) vertices by starting to add edges: For the first vertex, we are able to add \(n-2\) edges, for the second vertex, we are able to add \(n-3\) edges, etc. The total number of edges we can add to a graph of \(n-1\) vertices is equal to $\sum_{i=1}^{\ n-2} i $ which is equal to \( \frac{(n-1)(n-2)}{2}\). This presents a contradiction, as we assumed that \(m > \frac{(n-1)(n-2)}{2}\). Therefore, there cannot be a vertex in \(G\) with degree \(0\).

    
    \newpage
    \question{Bipartite Graph and Walks} 
    \begin{enumerate}
        \item For each vertex $z=z_1z_2\cdots z_n \in \{0,1\}^n$, consider the sum of the bits with even indices:
        \[E(z) = \sum_{j=1}^{n/2}z_{2j}.\]
        The vertex set of $G_n$ can be partitioned as $A\cup B$, where 
        \[A = \{z \in G_n: E(z) \text{ is even}\} \quad B = \{z \in G_n: E(z) \text{ is odd}\}.\]
        Now suppose that $x,y \in G_n$ are connected by an edge. Then, using the definition of the graph $G_n$ we may write $x=asb$ and $y=a\bar{s}b$ with $s \in \{01, 10\}$. Using the definition of $E$ and the form of $x$ and $y$ we see that
        \[E(x) - E(y) = \begin{cases}
            1, \quad &\text{if $a$ has even length,}\\
            -1, \quad &\text{if $a$ has odd length.}
        \end{cases}\]
        Therefore if $E(x)$ is even then $E(y)$ is odd and vice versa, and all edges of the graph $G_n$ have one endpoint in $A$ and one endpoint in $B$. Thus $G_n$ is bipartite.
        \item To show that $G_n$ is connected, note that two vertices $x\neq y \in V(G_n)$ are connected by an edge if and only if $x$ is obtained from $y$ by swapping two adjacent bits. Suppose we are given a binary string $\sigma \in V(G_n)$. Since $\sigma$ has $n/2$ ones, we can construct a walk from $\sigma$ to the string $1^{n/2}0^{n/2}$ by walking along edges that implement nearest-neighbour swaps to move the leftmost $1$ to the first bit, then using swaps of the bits $2, \dots, n$ to move the next $1$ to the second bit, and so on. Thus, given any two strings $\sigma, \sigma' \in V(G_n)$ we can construct a walk between them by concatenating the walk from $\sigma$ to $1^{n/2}0^{n/2}$ with the walk from $1^{n/2}0^{n/2}$ to $\sigma'$. This shows that $G_n$ is connected.
    \end{enumerate}

   
\end{enumerate}